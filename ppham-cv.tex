% Paul Pham's Curriculum Vitae

\documentclass[letter]{article}
\usepackage{fancyhdr}
\usepackage[osf]{mathpazo}
\usepackage[pdftex]{graphicx,color,hyperref}

\pagestyle{fancyplain}
\lhead[\fancyplain{}{\bfseries\thepage}]
    {\fancyplain{}{\bfseries\rightmark}}
\rhead[\fancyplain{}{\bfseries\leftmark}]
    {\fancyplain{}{Paul T. Pham \hfill\bfseries\thepage}}
\cfoot[]{}
\addtolength{\headheight}{1.6pt}

\addtolength{\topmargin}{-0.75in}	% repairing LaTeX's huge margins...
\addtolength{\textwidth}{.5in}	% same here....
\addtolength{\marginparwidth}{.1in}	% The word ``Memberships'' is really long below
\setlength{\headwidth}{\textwidth}
\setlength{\parindent}{0mm}	% indented paragraphs just don't make it
			   	% on a resume.  (My opinion; if you
			   	% disagree, give this whatever value you
			   	% want.) 
\setlength{\textheight}{9.5in}	% more margin hacking.  I have a large
			    	% resume to fit on one page.

\begin{document}

\thispagestyle{empty}           % no headers on the first page

\reversemarginpar		% this puts the margin notes on the
				% left-hand side of the resume.

{\LARGE {\bf Paul T. Pham}}
\par
\vspace{.25in}
E-mail: \href{mailto:ppham@cs.washington.edu}{ppham@cs.washington.edu}
\hspace*{\fill}
Dept. of Computer Science \& Engineering
\linebreak
Phone: (206) 859-0322
\hspace*{\fill}
University of Washington
\linebreak
Fax: (206) 543-2969
\hspace*{\fill}
Box 352350
\linebreak
\url{http://homes.cs.washington.edu/~ppham/}
\hspace*{\fill}
Seattle, WA 98195-2350

				% address stuff.  The \hspace*{\fill} is
				% necessary to get things laid out
				% correctly.  The sizes fed to \vspace are
				% variable, depending on how much
				% whitespace you want, and how much
				% stuff you're trying to fit on a page....
\par
\vspace{.25in}
				% if the \marginpar is at the beginning
				% of the line, it loses.  *sigh*.  This
				% is a typical entry in a resume.   (I'm
				% defining an entry as something
				% important enough to have a marginal note
				% setting it off.)  Give it some space
				% with \vspace, use \par to break
				% between paragraphs, enter your text,
				% and place the \marginpar at the end.

%%%%%%%%%%%%%%%%%%%%%%%%%%%%%%%%%%%%%%%%%%%%%%%%%%%%%%%%%%%%%%%%%%%%%%%%%%%%
{\bf University of Washington} \hspace*{\fill}September 2005---December 2006
\marginpar{{\bf Education}}
\par
Candidate for Doctor of Philosophy in Computer Science \hspace*{\fill}March 2010---Present\\
Expected graduation date June 2013.\\
\href{http://quantum.cs.washington.edu}{Quantum Computing Theory Group}\\
Advisor: Aram Harrow
\vspace{\baselineskip}
\par

{\bf Massachusetts Institute of Technology}
\par
Master of Engineering in Electrical Engineering \& Computer Science, February 2005.\\
Graduate GPA: 4.7/5.0\\
Thesis: \href{http://sourceforge.net/project/showfiles.php?group_id=129764&package_id=144780&release_id=307201}{A general-purpose pulse sequencer for quantum computing}.\\
Advisor: Isaac Chuang\\
\par
Bachelor of Science in Electrical Engineering and Computer Science, June 2004.\\
Undergraduate GPA: 4.2/5.0
\vspace{\baselineskip}
\par
				% This is a slightly more complicated
				% example.  \vspace gets a `doublespace'
				% (1.5\baselineskip) between entries, and
				% a  `singlespace' between paragraphs of
				% an entry.
				% (1.5\baselineskip means twice the value
				% of \baselineskip).  

%%%%%%%%%%%%%%%%%%%%%%%%%%%%%%%%%%%%%%%%%%%%%%%%%%%%%%%%%%%%%%%%%%%%%%%%%%%%%%
\vspace{\baselineskip}
\par
``Quantum compiling single-qubit gates with the Kitaev-Shen-Vyalyi procedure''
\marginpar{{\bf Publications}}\\
\textbf{P. Pham}\\
In preparation.

\vspace{\baselineskip}
\par
``A 2D nearest-neighbor quantum architecture for factoring''\\
\textbf{P. Pham}, K.M. Svore. \hfill \url{http://arxiv.org/abs/1207.6655}\\
Reversible Computation Workshop, June 2012 \hfill Copenhagen, Denmark\\

\vspace{\baselineskip}
\par
``Component-based Invisible Computing''\\
A. Forin, J. Helander, \textbf{P. Pham}, J. Rajendiran.\\
IEEE Realtime Embedded Systems Workshop, December 2001 \hfill 

%%%%%%%%%%%%%%%%%%%%%%%%%%%%%%%%%%%%%%%%%%%%%%%%%%%%%%%%%%%%%%%%%%%%%%%%%%%%%%
\vspace{\baselineskip}
\par
``A 2D Quantum Architecture for Factoring in Sub-Quadratic Depth''
\marginpar{{\bf Posters}}\\
\textbf{P. Pham}\\
Quantum Information Processing (QIP) Conference, December 2011.

\vspace{\baselineskip}
\par
``Quantum Compiling with Kitaev-Shen-Vyalyi''\\
\textbf{P. Pham}\\
Southwest Quantum Information and Technology (SQuInT), February 2011.

\vspace{\baselineskip}
\par
``Adiabatic Shelving to the $5D_{5/2}$ State in Trapped Barium Ions''\\
R. McClure, J. Booth, \textbf{P. Pham}, J. Wright, T. Noel, B. Blinov\\
Southwest Quantum Information and Technology (SQuInT), February 2011.

%%%%%%%%%%%%%%%%%%%%%%%%%%%%%%%%%%%%%%%%%%%%%%%%%%%%%%%%%%%%%%%%%%%%%%%%%%%%%%
\vspace{\baselineskip}

``Method and system for managing the execution of threads and data processing.''
\marginpar{{\bf Patents}}\\
A. Forin, J. Helander, \textbf{P. Pham}.\\
U.S. Patent Application No. 20030233392.\\
Filed on June 12, 2002.

\pagebreak

%%%%%%%%%%%%%%%%%%%%%%%%%%%%%%%%%%%%%%%%%%%%%%%%%%%%%%%%%%%%%%%%%%%%%%%%%%%%%%
\vspace{\baselineskip}
\par
{\bf {University of British Columbia}} \hfill Vancouver, Canada\marginpar{{\bf Invited}}
\par
{\em Quantum architecture, compiling, and 2D factoring} \hfill September 2012\marginpar{{\bf Talks}}
\par
Hosted by Robert Raussendorf.

\vspace{\baselineskip}
\par
{\bf {University of Innsbruck}} \hfill Innsbruck, Austria
\par
{\em Quantum architecture, compiling, and 2D factoring} \hfill July 2012
\par
Hosted by Rainer Blatt.

\vspace{\baselineskip}
\par
{\bf {University of Freiburg}} \hfill Freiburg, Germany
\par
{\em Quantum architecture, compiling, and 2D factoring} \hfill July 2012
\par
Hosted by Tobias Sch\"atz.

\vspace{\baselineskip}
\par
{\bf {University of Aarhus}} \hfill Aarhus, Denmark
\par
{\em Quantum architecture, compiling, and 2D factoring} \hfill July 2012
\par
Hosted by Michael Drewsen.

%%%%%%%%%%%%%%%%%%%%%%%%%%%%%%%%%%%%%%%%%%%%%%%%%%%%%%%%%%%%%%%%%%%%%%%%%%%%%%
\vspace{\baselineskip}
\par
{\bf {Pulse Programmer}} \hfill SourceForge\marginpar{{\bf Open Source}}
\par
{\em Project Admin, Lead Developer} \hfill January 2005---Present\marginpar{{\bf Experience}}
\par
\url{http://pulse-programmer.org}
\par
Built an open source reconfigurable radio-frequency signal generator
for quantum computing and quantum information processing experiments.

\vspace{\baselineskip}
\par
{\bf {Quantum Compiler}} \hfill SourceForge, Github
\par
{\em Project Admin, Lead Developer} \hfill January 2005---Present
\par
\url{http://quantum-compiler.org}
\par
Developed an open source code in Python and NumPy to implement the
Solovay-Kitaev quantum compiling algorithm for generic, multi-qubit gates
in SU(d). Simulated the Kitaev-Shen-Vyalyi quantum compiling algorithm in
QCL and wrote code to measure its required resources. Accepted as
qualifying examination project in the UW CSE Ph.D. program.

%%%%%%%%%%%%%%%%%%%%%%%%%%%%%%%%%%%%%%%%%%%%%%%%%%%%%%%%%%%%%%%%%%%%%%%%%%%%%%
\vspace{\baselineskip}
\par
{\bf {University of Washington Computer Science \& Engineering}} \hfill Seattle, WA\marginpar{{\bf Students}}
\par
{\bf Jeffrey Booth, Jr.} \hfill January 2010---Present\marginpar{{\bf Supervised}}
\par
{\bf Noah Siegel} \hfill September 2012---Present
\par
{\bf Andrea McCool} \hfill June 2010---Present
\par
{\bf Harshad Petwe} \hfill June 2010---August 2011
\par
{\bf Rob McClure} \hfill January 2010---March 2011
\par
{\bf John Williams} \hfill January 2010---May 2010
\par
{\bf David Nufer} \hfill January 2010---May 2010

%%%%%%%%%%%%%%%%%%%%%%%%%%%%%%%%%%%%%%%%%%%%%%%%%%%%%%%%%%%%%%%%%%%%%%%%%%%%%%
\vspace{\baselineskip}
\par
{\bf {Microsoft Research}} \hfill Seattle, WA\marginpar{{\bf Research}}
\par
{\em Research Intern} \hfill June---August 2011\marginpar{{\bf Experience}}
\par
\href{http://research.microsoft.com/en-us/groups/quarc/default.aspx}{Quantum Architectures and Computation Group}
\par
Mentor: Krysta Svore
\par
Designed a 2D nearest-neighbor quantum architecture for period-finding with
depth $O(L \log L)$ for factoring an $L$-bit integer. Pending patent
application.

\vspace{\baselineskip}
\par
{\bf {University of Washington Dept. of Physics and Astronomy}} \hfill Seattle, WA
\par
{\em Graduate Research Assistant} \hfill January---July 2007, May---June 2010
\par
\href{http://depts.washington.edu/qcomp/}{Trapped Ion Quantum Computing Group}
\par
Advisor: Prof. Boris Blinov
\par
Built a programmable radio-frequency system for ion trap control including
photomultiplier tube input counting.

\par

\vspace{\baselineskip}
\par
{\bf Max Planck Institute for Quantum Optics} \hfill Garching, Germany
\par
{\em Visiting Ph.D. Student} \hfill July 2005---August 2005
\par
\href{http://www.mpq.mpg.de/qsim/}{Quantum Analog Simulation Group}
\par
Advisor: Dr. Tobias Sch\"atz
\par
Built a programmable radio-frequency system for ion trap control with phase-coherent
frequency-switching.

\par

\vspace{\baselineskip}
\par
{\bf University of Innsbruck} \hfill Innsbruck, Austria
\par
{\em Visiting Ph.D. Student} \hfill February 2005---June 2005
\par
\href{http://heart-c704.uibk.ac.at/index.html}{Quantum Optics and Spectroscopy Group}
\par
Advisor: Univ. Prof. Rainer Blatt
\par
Built a programmable radio-frequency system for ion trap control with shaped amplitudes.
\par

\vspace{\baselineskip}
\par
{\bf MIT Center for Bits and Atoms} \hfill Cambridge, Massachusetts
\par
{\em Graduate Research Assistant} \hfill September 2003---January 2005
\par
\href{http://web.mit.edu/~cua/www/quanta/}{quanta Research Group}
\par
Advisor: Prof. Isaac Chuang
\par
Designed and built instrumentation for quantum computing experiments.
\par

\vspace{\baselineskip}
\par
{\bf Microsoft Research} \hfill Redmond, WA
\par
{\em Research Intern} \hfill June 2001---September 2001
\par
\href{http://research.microsoft.com/en-us/projects/mic/default.aspx}{Invisible Computing Group}\hfill June 2003---August 2003
\par
Mentors: Alessandro Forin, Johannes Helander
\par
Added work items to the scheduler of an embedded real-time kernel.
Designed and assembled the electronics for a wireless sensor demo.

\par

%%%%%%%%%%%%%%%%%%%%%%%%%%%%%%%%%%%%%%%%%%%%%%%%%%%%%%%%%%%%%%%%%%%%%%%%%%%%%%
\vspace{\baselineskip}

{\bf MIT ACM/IEEE Programming Competition} \hfill Cambridge, Massachusetts
\marginpar{{\bf Activities}}

{\em Contest Chair, Lead Developer, Organizer} \hfill 2001-2003
\vspace{0.5\baselineskip}
\par

\url{http://web.mit.edu/ieee/6.370/2003/web/}

\vspace{\baselineskip}

%%%%%%%%%%%%%%%%%%%%%%%%%%%%%%%%%%%%%%%%%%%%%%%%%%%%%%%%%%%%%%%%%%%%%%%%%%%%%%
{\bf University of Washington} \hfill Seattle, Washington
\marginpar{{\bf Teaching}}

\par
{\em Teaching Assistant, Computer Science \& Engineering Department}
\marginpar{{\bf Experience}}

\vspace{0.5\baselineskip}
\par
Advanced Internet Services (\href{http://www.cs.washington.edu/education/courses/454/12wi/}{CSE 454})\hfill January 2012---Present
\par
Professor Oren Etzioni

\vspace{0.5\baselineskip}
\par
The Hardware/Software Interface (\href{http://www.cs.washington.edu/education/courses/351/10sp/}{CSE 351})\hfill April---June 2010
\par
Professor Gaetano Borriello

\vspace{0.5\baselineskip}
\par
Data Structures (\href{http://www.cs.washington.edu/education/courses/326/06au/}{CSE 326})\hfill September---December 2006
\par
Professor Larry Snyder

\vspace{0.5\baselineskip}
\par
Software Development Tools (\href{http://www.cs.washington.edu/education/courses/303/06sp/}{CSE 303})\hfill April---June 2006
\par
Professor Magda Balazinska

\vspace{0.5\baselineskip}
\par
Algorithms (\href{http://www.cs.washington.edu/education/courses/417/06wi/}{CSE 417})\hfill January---March 2006
\par
Professor Larry Ruzzo

\vspace{0.5\baselineskip}
\par
Discrete Structures Class (\href{http://www.cs.washington.edu/education/courses/321/05au/}{CSE 321})\hfill September---December 2005
\par
Professors Dieter Fox \& Anna Karlin
\par

\vspace{\baselineskip}
\par
{\bf MIT Elec. Eng. \& Computer Science Dept.} \hfill Cambridge, Massachusetts

\vspace{0.5\baselineskip}
\par
{\em Teaching Assistant}
\par
Software Engineering Laboratory Class (\href{http://courses.csail.mit.edu/6.170/old-www/2004-Spring/admin-info/generalinfo.html#Staff}{6.170}) \hfill January 2004---May 2004
\par
Professor Rob Miller

\vspace{0.5\baselineskip}
\par
{\em Lab Assistant}
\par
Software Engineering Laboratory Class (6.170) \hfill September 2002---May 2003
\par
Professors Michael Ernst \& Daniel Jackson
\par

\vspace{\baselineskip}
\newpage

%%%%%%%%%%%%%%%%%%%%%%%%%%%%%%%%%%%%%%%%%%%%%%%%%%%%%%%%%%%%%%%%%%%%%%%%%%%%%%
{\bf Krysta Svore}
\marginpar{{\bf References}}\\
Researcher\\
Microsoft Research\\
1 Microsoft Way\\
Redmond, WA 98052\\
Phone: (425) 421-6996\\
E-mail: \url{ksvore@microsoft.com}\\

{\bf Aram Harrow}\\
Research Assistant Professor\\
University of Washington, Department of Computer Science and Engineering\\
Box 352350, Seattle, WA 98195-2350\\
Phone: (206) 616-0733\\
E-mail: \url{aram@cs.washington.edu}\\

{\bf Boris Blinov}\\
Associate Professor\\
University of Washington, Department of Physics and Astronomy\\
Box 351560, Seattle, WA 98195-1560\\
Phone: (206) 221-3780\\
E-mail: \url{blinov@uw.edu}\\

{\bf Tobias Sch\"atz}\\
Assistant Professor\\
Max Planck Institute for Quantum Optics\\
Hans-Kopfermann-Strasse 1\\
D-85748 Garching, Germany\\
Phone: +49-89-32905-199\\
E-mail: \url{tobias.schaetz@mpq.mpg.de}\\

{\bf Alessando Forin}\\
Principal Researcher\\
Microsoft Research\\
1 Microsoft Way\\
Redmond, WA 98052\\
Phone: (425) 936-1841\\
E-mail: \url{sandrof@microsoft.com}\\

{\bf Rainer Blatt}\\
University Professor of Physics\\
Institute for Experimental Physics\\
University of Innsbruck\\
Technikerstrasse 25/4\\
A-6020 Innsbruck\\
Austria\\
Phone: +43-512-507-6302\\
E-mail: \url{Rainer.Blatt@uibk.ac.at}\\

{\bf Isaac Chuang}\\
Professor\\
Departments of Physics and Electrical Engineering \& Computer Science\\
Massachusetts Institute of Technology\\
77 Massachusetts Ave\\
Boston, MA 02139\\
Phone: (617) 253-1692\\
E-mail: \url{ichuang@mit.edu}\\

%{\bf Doug Irvine}\\
%Software Development Manager\\
%Amazon.com\\
%701 5th Ave, Suite 1500, Seattle, WA 98104\\
%Phone: (206) 266-3154\\
%E-mail: \url{dirvine@amazon.com}\\

%{\bf Lindsay Michimoto}\\
%Graduate Program Advisor\\
%University of Washington, Department of Computer Science and Engineering\\
%Box 352350, Seattle, WA 98195-2350\\
%Phone: (206) 543-5758\\
%E-mail: \url{lindsaym@cs.washington.edu}\\

\end{document}




