\documentclass{article}

\title{Teaching Statement} \author{Paul Pham}

\begin{document}

\maketitle

You are a student sitting in a classroom in front
of your laptop along with thirty
of your classmates. This course is designed as a game with a series of
increasingly difficult levels which you can
unlock at your own pace.
%submitting correct homework for the current level
%before being able to proceed to the next one.
You spent several hours thinking
through Level 6 last night, before realizing you were stuck.
Rather than stay up all night stressing out,
you went to bed at a reasonable hour and are now well-rested. Some
other students have the course game website open on their computers, continuing
to work on their homework before class like you.

You turn to your neighbor and ask
``Hey, have you unlocked Level 6 about Turing machines yet?''

``Yeah, that one’s hard,” Andrea replies.
``You have to use a special symbol to mark the
end of the input. If you wait, he’ll probably cover it in lecture
today.'' Rather than
being cheating, this kind of peer collaboration is encouraged as
long as you submit your own work and acknowledge your contributors.

You can see that Andrea has finished all the required levels in the
class already (smarty pants!) and is taking an optional path.
``I really want to understand this
problem of P versus NP, and it'll connect with Level 1 of the complexity
class next quarter,'' she tells you.
%, ``but I already had a lot of this math
%background.'' This material is all new to you, however,
%and you are following the course lecture and notes pretty closely, unlocking
%each level week-by-week as the instructor covers the appropriate material.
Since there are no
deadlines, and all the levels are available to the class from the first
day, everyone works at their own pace. No one is ever bored, and the
incentive to cheat is vastly reduced.

Finally, I enter the room, saying hello to everyone and setting up my slides
in the few minutes before the bell rings. After some announcements, I read
out anonymous feedback from the last class and address them in class. Some of
these comments will result in changes, such as slowing down the course material,
while others will simply be acknowledged, such as requests to change the
final project goals. Then I open the floor for questions.
Some students raise their
hands and voice their confusion about a particular problem. Others are
curious how this material is relevant to their lives. I spend
about ten minutes attempting to answer the most popular or interesting questions,
before deferring the rest to after class or office
hours.

Then I give the lecture material for the next thirty minutes, which
should allow most students to complete Level 6 afterwards. This is followed
by a group brainstorming activity to come up with ways to simulate complex-seeming
behavior in real-life (like an ant solving a maze) using a Turing-machine.
Three student volunteers present their group's examples to the whole class.

``Okay, now it’s time for grading,'' I announce.
%``After you’ve graded two problems,
%you can get back to work on your current level.''
You and your
clasmates switch your website
into grading mode, and you choose to grade Mike’s Level 5
homework, which you completed last time.
%An answer key is provided on
%the screen next to Mike’s answers.
You pop up a chat window to Mike and
tell him one of his answers doesn't match the key, but he argues that his
alternative is correct.
As I’m walking around the room, you raise your hand, and I come
over to verify that indeed, Mike’s answer is also correct. You add a comment
to the answer key and allow Mike to proceed to Level 6.

After grading two other problems, you continue working on your own Level 6.
One of the problems involves creating your own question based on the course
material, and answering the original question of one of your classmates.
This encourages divergent, creative thinking, encourages personal interaction,
and gives me feedback about your particular interests.

At the end of class, I ask for everyone to switch to
feedback mode. You fill out a brief, anonymous survey
rating the time commitment of the class, the pace of new material, and suggesting
changes or new topics to me for next time.
You click submit, and the bell rings to end class.

The above scenario, while fanciful in some ways, describes my vision for an
ideal, social classroom environment supplemented by a
collaborative, online learning technology.
%A good teacher considers the
%student’s learning experience first and foremost, and then unfortunately
%must curtail it based on the realistic constraints of class time, personal
%attention, the available course materials, the teacher's knowledge, and other
%resources. However, these constraints can be mitigated by modern web technologies.

The scenario also demonstrates
my teaching philosophy, developed over two years of being a graduate TA,
culminating in designing and instructing the course
\emph{Quantum Computing for Beginners}
at the University of Washington. I approximate the learning experience
above using paper lecture notes and manual grading.
Why do all this work? My main motivation for teaching is that
it is fun for both me and my students. Furthermore, connecting with other
people and caring about our collaboration is the 
most effective way for me to both learn and produce research. I currently
mentor two advanced undergraduates in individual research projects related to
my thesis. Our meetings are work sessions, often full of long pauses and
backtracking and messy whiteboard sessions. I intend to continue this
practice as much as
possible as a faculty member.
Teaching myself, teaching others, and producing new work become inseparable.
%I often experience research and the pressure to
%publish papers as lonely and isolating, whereas
%connecting with students and caring about their education gives me energy.

One of my primary activities as a scientist and a teacher is to rekindle the curiosity
and feeling of playful exploration that I had as a child.
%As a teacher, I
%want to teach my students how to do the same.
This is the greatest skill I have learned
from my advisor, and from it all my other activities and results flow.
%At present, I mentor advanced undergrads in individual research projects,
%something I intend to keep doing in addition to classroom teaching.
%With these students, I discuss my process and work on a
%whiteboard as a means of crystallizing my own research.
I also have an innate interest in acting, public speaking, visual
design, photography, typography, effective communication, and wonder-making
as a means of transmitting excitement about the world. One of my great passions
is collecting amusing, fascinating, and striking examples of science
from ordinary experiences to illustrate concepts to my students; for example,
I use the fact that the number of spirals in pine cone seeds follows the
Fibonacci sequence, as an example of a physical, computational process in nature.

%For example,
%topics are often
%related in a complicated web of dependencies; I enjoy untangling this web into a
%linear narrative and making choices such as whether to start with a concrete
%example and abstract from it, or vice versa.

%As a child, I viewed learning as playful exploration, and I would like for my
%student. By building physical demonstrations with my hands,
%pursuing my curiosity, and by extension programming my computer, my
%education was self-guided and neverending, at the same time depending on
%the effort, creativity, and clear thinking of book authors, game creators, television
%show producers, and many other people. I would like 

I am currently interested in continuing to develop
my class on quantum computing, perhaps through an external publisher.
However, I have diverse interests in teaching
other classes about recent technology: machine learning, programming
mobile devices, and using depth cameras such as the Kinect for user interfaces.
I am also interested in exploring the entrepreneurial aspects of these
technologies by helping students to sell their apps in public marketplaces or
connecting them with other business development resources.
While I am by no means an expert on the above topics, I have a
rich social network of experts from graduate school to advise me, a willingness
to devote time to exploring existing online resources, and a strong
desire to learn this material myself.
%My
%fresh perspective, enthusiasm, and willingness to learn more than compensate for
%this inexperience.

I have three main teaching techniques which distinguish me from most other
teachers: storytelling, personal involvement, and constantly responding to feedback.
I have enjoyed writing narratives since I wrote childhood bedtime stories for my
sister, and even the simplest of characters and plotlines make a huge difference
to me in my interest in solving a problem. The history of computer
science, and especially quantum physics, is a very rich goldmine of strong
personalities and world-changing events (e.g. the development of the
nuclear bomb in World War II) that provide my current class with a compelling
story. By personal involvement, I mean that I try to do everything that I expect
my students to do. If they are working on a final project, I will choose a
project of similar difficulty and do it with them, both as a means of empathizing
with their experience and discovering potential pitfalls and responses along the
way. I am also constantly collecting feedback and responding to it flexibly.
This also reduces that amount of advanced planning I have to do; too much
planning will deaden my presentation of the material, as well as leaving no
room for student participation and input.

%to use it o fand use it to adjust the pace
%of my class, my coverage of future topics, and my choice 
%My lecture notes are a mix of
%historical fiction and comic-book-style illustrations,
%in which the student interacts with famous
%scientists in the 1930s through a protagonist, learning technical scientific
%content and equations, and solving homework problems posed as
%collaborative research or puzzles.

%Games are an important manifestation of storytelling, and one that I would
%like to pursue in future, cross-disciplinary collaborations.
%As an undergraduate I created a highly successful Java programming competition
%to teach students about artificial intelligence using a video game engine.
%This also generated mutually profitable
%partnerships with large software companies, although eventually the cash
%prizes became so substantial that I became disillusioned by the contest.
%Treading this ambiguity between real-world relevance and learning for its own
%sake.

%My teaching techniques in the classroom start with traditional lectures
%interspersed with group activities, such as student grading and discussion of
%homework. In a recent class, students brainstormed examples from their
%personal lives or the physical world of analog computers, probabilistic
%mechanisms, and other computational resources that occur in nature.
%I will often forward relevant articles on popular news
%sites like Reddit, Ars Technica, and Slashdot about quantum computing along
%with additional commentary to show
%them that the course material is cutting edge, part of a dynamic world,
%and likely to be important in the future.

%I experiment and improvise
%heavily in my classes; after trying to flip the class entirely so that
%homework is done and graded in-class and lecture notes are read outside
%of class, I’ve settled on a half-lecture, half-in-class exercises
%format. This feels the most satisfactory to both me and my students.
%Another technique that I am a strong proponent of is letting students
%make up one homework question, posting it to the class message board,
%and then answering the question posed by another student. This teaches
%divergent, creative thinking, rather than the convergent thinking which
%usually only leads to a single correct answer. It also gives me insight
%into topics that are of interest to students, which helps guide my
%future classes. For example, I was only planning to spend one or two
%classes on the Bloch sphere, which is a geometric / pictorial
%representation of a quantum bit, but based on student questions, I chose
%to spend an extra class. I removed later, more advanced topics in favor
%of providing a firmier foundation for student curiosity. Other popular
%teaching tools: I invite a few guest lecturer who are more expert on
%certain topics than I am, and this helps me learn as well.

%I dedicate
%time for one-on-one office hours and make timeslots available to
%students, rather than asking them to schedule individually with me,
%which almost no student has ever done except under extreme duress.
%In
%the past, I’ve tried incentives such as giving bonus points for sharing
%techniques with classmates, but this felt forced and admittedly, a
%labor-saving measure from myself to minimize my understanding of the
%problems students were going through. The best incentive is to show
%students that you care. As an instructor, this  means actually doing the
%homework before (ideally) or at least during your assignment to the
%students. As a TA, this often means doing lab work with them. Also,
%creating field trips to show students the material in action in the real
%world. In my current class, I arranged for them to take a lab tour
%through two experimental groups in the physics department who are
%actively trying to implement a physical quantum computing device using
%two different technologies: trapped atomic ions and the nitrogen-vacancy
%center in diamonds.

In teaching there is a rich landscape of feedback to gauge one's effectiveness,
both explicity and implicitly collected. In addition to the anonymous
feedback forms, I regularly solicit external observers in my class (and
volunteer to observe my fellow instructors' classes), and in the past I have
recorded videos of my lectures as the most ruthless means of feedback.
This is perhaps the only way to improve verbal delivery,
vary pacing effectively, eliminating verbal ticks, and maintaining eye contact.
Being
able to teach effectively through videos is a vital skill that I would like to
cultivate for producing future online courses.
Other implicit forms of
feedback are just as vital: the attendance of students in the class and
office hours, 
the submission rate of the homework, and the quality of their questions.
For example, do they catch my unintentional mistakes in the lecture material?
%I have also trained myself to become very attentive to the attitudes of
%students while in class, but not to be controlled by them.
My most rewarding moments as a teacher have come from positive student feedback.
One student asked for more homework to solidify their understanding of a topic that I
presented too quickly in class. Another student asked for optional problems to
test more advanced material.

%Every week, students fill out
%short paper surveys, which is my main feedback mechanism. In the past, I've
%used both
%external observers in my class and video recordings to give me feedback.
%I've found video to be the most immediate, ruthless form of objective feedback.
%In the future, I would like to record all my lectures, not just for
%an online learning course, but to improve the
%delivery of my words, my pacing, my verbal ticks, how much I talk to the
%screen instead of to my students.




%So far, I have addressed ways that I have created a customized, dynamic
%learning experience for my students, often by creating a course from
%scratch. However, I anticipate two main objections to this approach: it
%is often not feasible to do this, and existing course material should
%definitely be used if possible. However, since my main research plan in
%the future is the educational use of games, this necessitates creating
%much of the course content from scratch. Furthermore, this is the most
%interesting part of teaching for me, and I would heavily prioritize this
%duty The other is that my approach of personal attention will not scale
%to a class of hundreds of students, which is often the case of “service
%classes” like an introduction to programming. Although I do not have
%much experience with teaching such a large class, I would be delighted
%to adapt my philosophy and approach to my army of TAs, who I hope can
%devote personal attention, and by proxy to their students in sections of
%30 or fewer, much in the same way that I would to my own students.

\end{document}